\documentclass[a4paper,11pt]{article}
\usepackage{color}
\usepackage{amsmath,amssymb,amsthm}
\usepackage{amsfonts}
%\usepackage{mathrsfs}
\usepackage{eucal}
\numberwithin{equation}{section}


\begin{document}
\pagestyle{empty}

% 表紙部分
\begin{center}
  {\LARGE\bfseries Semi-Classical Analysis and Spectral Theory of Schrödinger Operators on Non-Compact Spaces}\\[1ex]


  \textbf{Date}: September 3--5, 2025\\
  \textbf{Venue}: RIMS, Room 420
\end{center}

\vspace{1ex}

\section*{Program}

\subsection*{Day 1: September 3}
\begin{tabbing}
  \hspace{3cm}\=\kill
  13:00--14:00 \> \textbf{Serge Richard}  (Nagoya University)\\
                \> \emph{Spectral and scattering theory in a discrete setting 1}\\

  14:10--15:10 \> \textbf{Setsuro Fujiie}  (Ritsumeikan University)\\
                \> \emph{Semiclassical analysis for matrix-valued operators
and applications to spectral}\\
               \> \emph{ asymptotics 1}\\[1ex]

  15:40--16:40 \> \textbf{Serge Richard}  (Nagoya University)\\
                \> \emph{Spectral and scattering theory in a discrete setting 2}
\end{tabbing}

\subsection*{Day 2: September 4}
\begin{tabbing}
  \hspace{3cm}\=\kill
  9:30--10:30 \> \textbf{Setsuro Fujiie}  (Ritsumeikan University)\\
                \> \emph{Semiclassical analysis for matrix-valued operators
and applications to spectral}\\
               \> \emph{ asymptotics 2}\\

  10:40--11:40 \> \textbf{Shota Fukushima}  (Gunma University)\\
                \> \emph{Mathematical analysis of transmission problem across ideal and resistive}\\
               \> \emph{ interface 1}\\

  13:00--14:00 \> \textbf{Setsuro Fujiie}  (Ritsumeikan University)\\
                \> \emph{Semiclassical analysis for matrix-valued operators
and applications to spectral}\\
               \> \emph{ asymptotics 3}\\

  14:40--15:40 \> \textbf{Kentaro Kameoka}  (Ritsumeikan University)\\
                \> \emph{Semiclassical shape resonances for Stark and magnetic Stark Hamiltonians 1}\\
                
  14:40--15:40 \> \textbf{Setsuro Fujiie}  (Ritsumeikan University)\\
                \> \emph{Semiclassical analysis for matrix-valued operators
and applications to spectral}\\
               \> \emph{ asymptotics 4}
\end{tabbing}

\subsection*{Day 3: September 5}
\begin{tabbing}
  \hspace{3cm}\=\kill
  9:30--10:30 \> \textbf{Shota Fukushima}  (Gunma University)\\
                \> \emph{Mathematical analysis of transmission problem across ideal and resistive}\\
               \> \emph{ interface 2}\\

  10:40--11:40 \> \textbf{Kentaro Kameoka}  (Ritsumeikan University)\\
                \> \emph{Semiclassical shape resonances for Stark and magnetic Stark Hamiltonians 2}
\end{tabbing}

\newpage

\section*{Titles and Abstracts}

\subsection*{\textbf{Setsuro Fujiie}  (Ritsumeikan University)}
\textbf{Title}: Semiclassical analysis for matrix-valued operators
and applications to spectral asymptotics\\
\textbf{Abstracts}: In this course we consider Hermitian matrix-valued semiclassical pseudo-differential operators, mainly $2\times 2$ matrix-valued operator with two Schr\"odinger operators on the diagonal and small of order h interactions on the off-diagonal (called {\it coupled Schr\"odinger operator} in the following). Each eigenvalue of the matrix-valued symbol defines a Hamiltonian system. We focus on the problem arising from a {\it crossing} of two or more Hamiltonian flows.

The course consists of three parts.

The first part is devoted to the basic theory of semiclassical
pseudodifferential operators. The goal of this part is a theorem of propagation of semiclassical wave front set. The real principal type condition in the well known scalar case is generalized here to the {\it microhyperbolic} condition in the sense of Ivrii.
A crossing point of Hamiltonian flows is microhyperbolic if there exists a scalar function increasing along these flows near the crossing point.

From the second part, we restrict ourselves, for more detailed study, 
to a one-dimensional coupled Schr\"odinger operator. We study the semiclassical asymptotic behavior of the solutions microlocally near a crossing point. It is expressed as what we call {\it microlocal scattering
matrix}, which relates the WKB solutions on the incoming Hamiltonian flows
with the ones on the outgoing flows. We will see that the first term of the semiclassical asymptotic expansion of this matrix is an identity and that the second term is off-diagonal, corresponding to the change of flows, of order $h^{\frac 1{m+1}}$ where $m$ is the contact order of the crossing.

In the last part, we apply the previous studies to the semiclassical
asymptotics of eigenvalues or resonances of a coupled Schroedinger operator with crossings. The Hamiltonian flows make a directed graph 
when we regard the crossing points as vertices. The WKB solutions on each edge (Hamiltonian flow) together with the microlocal scattring matrices at each vertiex enable us to define a monodromy matrix of the microlocal solutions on the graph.
Eigenvalues or resonances are then characterized in terms of this matrix.

\subsection*{\textbf{Shota Fukushima}  (Gunma University)}
\textbf{Title}: Mathematical analysis of transmission problem across ideal and resistive interface\\
\textbf{Abstracts}: We consider electrostatic or thermal systems with conductors placed in Euclidean space. Conductors have three main characteristics to be controlled: shape, conductivity and interface resistance. If the conductivity is locally constant, then the system is modelled by the Laplace equation, with these three characteristics incorporated into the transmission conditions. In this talk, we deal with two types of conductors. One is a conductor with negative conductivity (plus small imaginary part) and ideal (non-resistive) interface, which motivates the spectral analysis of Neumann-Poincaré operator. The other is a perfect conductor with constant interface resistance. In this case, the solution exhibits distinct behavior against the ideal interface case, although these two cases are connected by the limit as the interface resistance tends to zero. This talk is based on joint work with Yong-Gwan Ji (KIAS), Hyeonbae Kang (Inha University), Xiaofei Li (Zhejiang University of Technology) and Yoshihisa Miyanishi (Shinshu University). 

\subsection*{\textbf{Kentaro Kameoka}  (Ritsumeikan University)}
\textbf{Title}: Semiclassical shape resonances for Stark and magnetic Stark Hamiltonians\\
\textbf{Abstracts}: We consider resonances of Stark and magnetic Stark Hamiltonians. We explain the definition of resonances through the complex distortion of  Hamiltonians. Then we assume that there are potential wells and study the semiclassical behavior of resonances which are exponentially close to the real axis. The magnetic Stark Hamiltonian part is based on joint work with Naoya Yoshida.  


\subsection*{\textbf{Serge Richard}  (Nagoya University)}
\textbf{Title}: Spectral and scattering theory in a discrete setting\\
\textbf{Abstracts}: During these two presentations, we shall start by recalling the
framework of topological crystals, which corresponds to the most general
$\mathbb{Z}^d$ periodic discrete setting. The spectral analysis of the Laplace
operator or of the adjacency matrix will then be briefly sketched, and
the scattering theory for perturbations of these operators will be
considered. Perturbed systems are allowed to have an infinite number of
new edges, as long as suitable decay conditions are imposed. In the
final part of the presentation, discrete time evolution groups will also
be introduced and various concepts of scattering theory will be further
illustrated. 



\end{document}